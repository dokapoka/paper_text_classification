\documentclass{article} 
\usepackage[utf8]{inputenc}
\usepackage[T2A]{fontenc}
\usepackage[russian, english]{babel}
\usepackage{graphicx}	
\usepackage{amsmath, amssymb}
\usepackage{titlesec}
\newcommand\tab[1][1cm]{\hspace*{#1}}
\usepackage[14pt]{extsizes}
\usepackage{setspace,amsmath}
\usepackage[left=20mm, top=15mm, right=15mm, bottom=15mm, nohead, footskip=10mm]{geometry} 
\usepackage{mathtext}
\usepackage[dvipsnames]{xcolor}
\usepackage{ragged2e}
\sloppy
\usepackage{caption}
\usepackage{hyperref}

\hypersetup{
    colorlinks=true,
    linkcolor=black,
    urlcolor=blue
}

\graphicspath{ {img/} }
%--------------------------------------------------------------------
\begin{document}                                                         
\newpage
\thispagestyle{empty}

\begin{center}
Федеральное государственное образовательное бюджетное учреждение\\ высшего профессионального образования\\ 
\textbf{«Финансовый университет при Правительстве Российской Федерации»}\\
\end{center}
	
\vspace{2em}
	
\begin{center}
\textbf{Департамент анализа данных, принятия решения и финансовых технологий}\\ 
\end{center}
	
\vspace{2em}
	
\begin{center}
\textbf{Курсовая работа\\
\vspace{3mm}}
по дисциплине \textbf{"Технологии анализа данных и машинное обучение"} на тему: \\ \vspace{2em} \textbf{"Классификация текстов методами машинного обучения"}\\
\vspace{3mm}
Вид исследуемых данных: Корпус новостей с сайта Lenta.Ru\\
\end{center}
	
\vspace{6em}
		
\begin{flushright}
Выполнила:\\
студентка группы ПМ17-1\\
Баданина Н. Д.\\
Научный руководитель:\\
доктор техн. наук, доцент\\
Судаков Владимир Анатольевич
\end{flushright}
	
\vspace{\fill}
	
\begin{center}
Москва 2020
\end{center}

%--------------------------------------------------------------------
\newpage
	
\renewcommand*\contentsname{Содержание}
\makeatletter
\renewcommand{\l@section}{\@dottedtocline{1}{0em}{2em}}
\renewcommand{\l@subsection}{\@dottedtocline{1}{0em}{2.6em}}
\renewcommand{\l@subsubsection}{\@dottedtocline{1}{0em}{3.2em}}

\makeatother
\setcounter{page}{2}

\begin{center}
	\tableofcontents
\end{center}
%--------------------------------------------------------------------
\newpage
\addcontentsline{toc}{section}{Введение}
\section*{Введение}
%--------------------------------------------------------------------
\newpage
\subsection{Сложности при обработке текстов на естественном языке}
%--------------------------------------------------------------------
\newpage
\subsection{Перспективы развития технологии}
%--------------------------------------------------------------------
\newpage
\section{Теоретическая часть} 
%--------------------------------------------------------------------
\newpage
\subsection{Задача классификации текстов}
%--------------------------------------------------------------------
\newpage
\subsection{Модели классификации}
%--------------------------------------------------------------------
\newpage
\section{Практическая часть}
\subsection{Работа с данными}
%--------------------------------------------------------------------
\newpage
\addcontentsline{toc}{section}{Заключение}
\section*{Заключение}
%--------------------------------------------------------------------
\newpage
\addcontentsline{toc}{section}{Список использованных источников}
\section*{Список использованных источников}
\renewcommand{\refname}{}
\begin{thebibliography}{7}
\bibitem{i1} Applied Text Analysis with Python by Benjamin Bengfort, Rebecca Bilbro, and Tony Ojeda (O’Reilly). 978­1­491­96304­3
\end{thebibliography}
%--------------------------------------------------------------------
\newpage
\addcontentsline{toc}{section}{Приложение}
\section*{Приложение}
\tabИсходный код можно найти по \href{https://github.com/dokapoka/paper_text_classification}{ссылке}.





 \end{document}  
