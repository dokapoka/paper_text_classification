\documentclass{article} 
\usepackage[utf8]{inputenc}
\usepackage[T2A]{fontenc}
\usepackage[russian, english]{babel}
\usepackage{graphicx}	
\usepackage{amsmath, amssymb}
\usepackage{titlesec}
\newcommand\tab[1][1cm]{\hspace*{#1}}
\usepackage[14pt]{extsizes}
\usepackage{setspace,amsmath}
\usepackage[left=20mm, top=15mm, right=15mm, bottom=15mm, nohead, footskip=10mm]{geometry} 
\usepackage{mathtext}
\usepackage[dvipsnames]{xcolor}
\usepackage{ragged2e}
\sloppy
\usepackage{caption}
\usepackage{hyperref}

\hypersetup{
    colorlinks=true,
    linkcolor=black,
    urlcolor=blue
}

\graphicspath{ {img/} }
%--------------------------------------------------------------------
\begin{document}                                                         
\newpage
\thispagestyle{empty}

\begin{center}
Федеральное государственное образовательное бюджетное учреждение\\ высшего профессионального образования\\ 
\textbf{«Финансовый университет при Правительстве Российской Федерации»}\\
\end{center}
	
\vspace{2em}
	
\begin{center}
\textbf{Департамент анализа данных, принятия решения и финансовых технологий}\\ 
\end{center}
	
\vspace{2em}
	
\begin{center}
\textbf{Курсовая работа\\
\vspace{3mm}}
по дисциплине \textbf{"Технологии анализа данных и машинное обучение"} на тему: \\ \vspace{2em} \textbf{"Классификация текстов методами машинного обучения"}\\
\vspace{3mm}
Вид исследуемых данных: Корпус новостей с сайта Lenta.Ru\\
\end{center}
	
\vspace{6em}
		
\begin{flushright}
Выполнила:\\
студентка группы ПМ17-1\\
Баданина Н. Д.\\
Научный руководитель:\\
доктор техн. наук, профессор\\
Судаков Владимир Анатольевич
\end{flushright}
	
\vspace{\fill}
	
\begin{center}
Москва 2020
\end{center}

%--------------------------------------------------------------------
\newpage
	
\renewcommand*\contentsname{Содержание}
\makeatletter
\renewcommand{\l@section}{\@dottedtocline{1}{0em}{2em}}
\renewcommand{\l@subsection}{\@dottedtocline{1}{0em}{2.6em}}
\renewcommand{\l@subsubsection}{\@dottedtocline{1}{0em}{3.2em}}

\makeatother
\setcounter{page}{2}

\begin{center}
	\tableofcontents
\end{center}
%--------------------------------------------------------------------
\newpage
\addcontentsline{toc}{section}{Введение}
\section*{Введение}
\tabНа сегодняшний день технологии развиваются с экспоненциальной скоростью. В научном мире появилась целая новая область знаний, которая требует изучения - это Искусственный Интеллект (ИИ). В ИИ, как подраздел можно включить машинное обучение и его алгоритмы. Одним из примеров алгоритмов машинного обучения являются нейронные сети. Пик развития машинного обучения начался ориентировочно с 2015 года, когда началась активная цифровизация, внедрение современных цифровых технологий, бизнесов, уход в онлайн. Это подтолкнуло компании к вложению средств в изучение области ИИ.\\
\tabВ данной работе речь пойдет об алгоритмах обработки текста на естественном языке. Голосовые помощники, чат-боты, умные устройства для дома позволяют компаниям привлечь дополнительную прибыль. Технологии, основанные на распознавании естественных языков создают новый интерфейс для взаимодействия с пользователем. Таким образом, создается эффект геймификации, что увеличивает возврат клиентов (retention) и уменьшает отток (churn).\\
\tabНа Российском рынке умные колонки с голосовыми помощниками стоят в малом количестве домохозяйств. Этот рынок развит в США, но имеет большой потенциал и в странах СНГ. На данный момент чат-боты используются не столько для увеличения продаж, сколько для уменьшения операционный затрат. К примеру, в банковском мобильном приложении можно задать вопрос в чат и ответит не оператор, а бот. При этом, компания экономит на затратах на операторов.\\
\tabПримером внедрения анализа естественного языка може служить поддержка "тегов рекомендаций", реализованная, к примеру, Netflix, YouTube. Тег - метоинформация о фрагменте контента важная для поиска и рекомендации. Теги определяют свойства описываемого ими контента и могут использоваться для группировки схожих элементов и предложения описательных названий для таких групп.\\
В речевом анализе аудиоданные преобразуются в текст, к которому можно применить алгоритмы NLP.
%--------------------------------------------------------------------
\newpage
\subsection{Сложности при обработке текстов на естественном языке}
\tabОсновной сложностью при обработке текстов на естественном языке средствами языков программирования является понимание алгоритмом контекста, в рамках которого идет обработка отдельного слова. Зачастую в тексте используются слова в переносном значении или в значении, которое установили собеседники между собой по договоренности. При существовании множества смыслов язык должен быть изботочен. Избыточность является серьбезной проблемой при построении алгоритмов NLP так как разработчики не могут и не будут указывать буквальный смысл каждого ассоциативного слова. Единицей анализа текста является лексема - строка кодированных байтов, представляющая собой текст. Лексема "батарея" изменила свой смысл с течением времени. Так, в текстах 19 века и позже можно увидеть это слово для обозначения артиллерийского подразделения из нескольких орудий. В современных публикациях лексема используется для обозначения хранилища, преобразующего химическую энергию в электронную.
%--------------------------------------------------------------------
\newpage
\subsection{Перспективы развития технологии}
Развитие технологии классификации текстов началось с введения спам-фильтров для почты. Приложения, основанные на использовании естественного языка только начинают распространяться, но в будущем могут взять на себя задачи, которые сейчас решаются стандартными формами и  интерфейсами.
%--------------------------------------------------------------------
\newpage
\section{Теоретическая часть} 
%--------------------------------------------------------------------
\newpage
\subsection{Задача классификации текстов}
\tabЦелью машинного обучения является подгонка существующих данных под некоторую модель, создание представления реального мира, помогающего прини­мать решения или генерировать прогнозы на основе новых данных, путем поиска закономерностей в них. На практике для этого выбирается семейство моделей, определяющих связи между целевыми и входными данными, задается форма, включающая параметры и особенности, а затем с помощью некоторой оптимизации минимизируется ошибка модели на обучающих данных. Затем обученной модели можно передавать новые данные, на сонове которых она будет строить прогноз и возвращать метки, вероятности, признаки принадлежности или значения. Задача состоит в том, чтобы найти баланс между свособностью с высокой точность. находить закономерности в известных данных и способностью обобщени для анализа данных, которые модель не видела прежде. Многие приложения для анализа естественного языка включают не одну, а целое множество моделей машиноого обучения, взаимодействующий между собой и влияющих друг на друга. Модели могут повторно обучаться на новых данных, нацеливаться на новые пространства решений.
%--------------------------------------------------------------------
\newpage
\subsection{Модели классификации}
%--------------------------------------------------------------------
\newpage
\section{Практическая часть}
\subsection{Работа с данными}
%--------------------------------------------------------------------
\newpage
\subsection{Инструменты разработки}
%--------------------------------------------------------------------
\newpage
\addcontentsline{toc}{section}{Заключение}
\section*{Заключение}
%--------------------------------------------------------------------
\newpage
\addcontentsline{toc}{section}{Список использованных источников}
\section*{Список использованных источников}
\renewcommand{\refname}{}
\begin{thebibliography}{7}
\bibitem{i1} Applied Text Analysis with Python by Benjamin Bengfort, Rebecca Bilbro, and Tony Ojeda (O’Reilly). 978­1­491­96304­3
\end{thebibliography}
%--------------------------------------------------------------------
\newpage
\addcontentsline{toc}{section}{Приложение}
\section*{Приложение}
\tabИсходный код можно найти по \href{https://github.com/dokapoka/paper_text_classification}{ссылке}.





 \end{document}  